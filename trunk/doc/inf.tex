\documentclass[titlepage]{article}
\usepackage[spanish]{babel}
\usepackage[pdftex]{hyperref}
\usepackage[pdftex]{graphicx}

\bibliographystyle{plain}

\title{CC51D: Proyecto: Mail Snatcher \\Informe de avance}
\author{Felipe Ignacio Ca\~nas Sabat \and Pablo Ignacio Sep\'ulveda Rojas
\and Prof: Alejandro Hevia}

\begin{document}
\maketitle
\section{Introducci\'on}
El proyecto denominado MailSnatcher, consiste en capturar y desplegar los
correos electr\'onicos que sean enviados usando un canal inseguro en una red
wireless.
\section{Funcionalidades Esperadas}
El sistema, debe ser capaz de desplegar los mails le\'idos usando la mayor
cantidad de protocolos posibles. En particular se pretende poder leer en orden:
\begin{itemize}
\item POP
\item SMTP
\item IMAP
\item HTTP (hotmail y gmail)
\end{itemize} 
El orden original se decidi\'o por la dificultad de comprender cada protocolo. 
Se est\'a considerando intentar leer hotmail y gmail antes que SMTP e IMAP,
pues son m\'as usados, y generar'ian un mayor impacto.
\section{Dise\~no Preliminar}
Se pretende usar un dise\~no orientado a objetos. Existe un primer objeto
Capturador, que interactuar\'a con pcap y su funci\'on ser\'a obtener los
paquetes enviados (filtrados mediante el puerto, as\'i 110 ser\'a POP, 80
ser\'a HTML, etc) y armar un objeto Mensaje, el que contendr\'a la IP y el
puerto tando del emisor como del receptor (estos son los identificadores).
Desde el objeto Mensaje se heredar\'an MensajePop, MensajeHTTP, y as\'i para
cada uno de los protocolos que se implementen.
\section{Trabajo realizado}
Hasta ahora, se ha estado investigando los protocolos y la librer\'ia y 
tambi\'en desarrollando.
\subsection{Investigaci\'on}
Se probaron las librer\'ias jpcap y jnetpcap para hacer el desarrollo en Java,
\'estas hacen uso de pcap mediante JNI, y lamentablemente no funcionan para 
procesadores de 64 bits (que es el equipo con que se cuenta), por lo que 
finalmente se decidi\'o usar simplemente pcap y desarrollar en c++.

Adem\'as se investig\'o principalmente el protocolo POP, adem\'as se investig'o
el protocolo HTTP, para dimensionar las dificultades involucradas. Adem\'as se
revisaron los protocolos ethernet, IP y TCP, para poder obtener la
informaci\'on relevante del paquete.
\subsection{Desarrollo}
Hasta ahora, se ha desarrollado gran parte del capturador, se prob\'o
capturando paquetes POP y funciona. Despliega la IP, puerto (tanto del enviador
como del receptor) y el contenido del mensaje (el comando o la respuesta).

A continuaci\'on un ejemplo de captura de una sesi\'on POP.
\begin{verbatim}
src=c0a8010a dst=c0a8010f
src=50010 dst=110
src=c35a dst=6e
src=c0a8010f dst=c0a8010a
src=110 dst=50010
src=6e dst=c35a
src=c0a8010a dst=c0a8010f
src=50010 dst=110
src=c35a dst=6e
src=c0a8010f dst=c0a8010a
src=110 dst=50010
src=6e dst=c35a

+OK Dovecot ready.
src=c0a8010a dst=c0a8010f
src=50010 dst=110
src=c35a dst=6e
src=c0a8010a dst=c0a8010f
src=50010 dst=110
src=c35a dst=6e
user acanas
src=c0a8010f dst=c0a8010a
src=110 dst=50010
src=6e dst=c35a
src=c0a8010f dst=c0a8010a
src=110 dst=50010
src=6e dst=c35a

+OK
src=c0a8010a dst=c0a8010f
src=50010 dst=110
src=c35a dst=6e
src=c0a8010a dst=c0a8010f
src=50010 dst=110
src=c35a dst=6e
pass 1234
src=c0a8010f dst=c0a8010a
src=110 dst=50010
src=6e dst=c35a

+OK Logged in.
src=c0a8010a dst=c0a8010f
src=50010 dst=110
src=c35a dst=6e
src=c0a8010a dst=c0a8010f
src=50010 dst=110
src=c35a dst=6e
list
src=c0a8010f dst=c0a8010a
src=110 dst=50010
src=6e dst=c35a

+OK 2 messages:
1 861
2 1057
.
src=c0a8010a dst=c0a8010f
src=50010 dst=110
src=c35a dst=6e
src=c0a8010a dst=c0a8010f
src=50010 dst=110
src=c35a dst=6e
retr 1
src=c0a8010f dst=c0a8010a
src=110 dst=50010
src=6e dst=c35a

+OK 861 octets
Return-path: <fcanas@Quasimodo.casa>
Envelope-to: acanas@Quasimodo.casa
Delivery-date: Mon, 12 Jan 2009 00:43:25 -0300
Received: from fcanas by localhost with local (Exim 4.63)
(envelope-from <fcanas@Quasimodo.casa>)
id 1LMDi1-0004z8-DZ
for acanas@Quasimodo.casa; Mon, 12 Jan 2009 00:43:25 -0300
Date: Mon, 12 Jan 2009 00:43:25 -0300
To: Alberto Ignacio Canas Sabat <acanas@Quasimodo.casa>
Subject: Hola
Message-ID: <20090112034325.GA19153@Quasimodo.casa>
MIME-Version: 1.0
Content-Type: text/plain; charset=us-ascii
Content-Disposition: inline
User-Agent: Mutt/1.5.13 (2006-08-11)
From: Felipe Ignacio Canas Sabat <fcanas@Quasimodo.casa>
X-SA-Exim-Connect-IP: <locally generated>
X-SA-Exim-Mail-From: fcanas@Quasimodo.casa
X-SA-Exim-Scanned: No (on localhost); SAEximRunCond expanded to false

Bienvenido.

Solo de prueba.

-Felipe
.
src=c0a8010a dst=c0a8010f
src=50010 dst=110
src=c35a dst=6e
src=c0a8010a dst=c0a8010f
src=50010 dst=110
src=c35a dst=6e
quit
src=c0a8010f dst=c0a8010a
src=110 dst=50010
src=6e dst=c35a

+OK Logging out.
src=c0a8010a dst=c0a8010f
src=50010 dst=110
src=c35a dst=6e
src=c0a8010f dst=c0a8010a
src=110 dst=50010
src=6e dst=c35a
\end{verbatim}
Hay algunos paquetes que no tienen informaci\'on (los ack por ejemplo) los que
ser\'an descartados por el objeto Mensaje.
\section{Trabajo Restante}
Lo siguiente ser\'a crear el Objeto mensaje y la estructura que organizar\'a el
tr'afico en conversaciones. Luego se har\'a el MensajePOP, para mostrar la
informaci\'on relevante.
\end{document}
